\documentclass[]{article}
\usepackage{lmodern}
\usepackage{amssymb,amsmath}
\usepackage{ifxetex,ifluatex}
\usepackage{fixltx2e} % provides \textsubscript
\ifnum 0\ifxetex 1\fi\ifluatex 1\fi=0 % if pdftex
  \usepackage[T1]{fontenc}
  \usepackage[utf8]{inputenc}
\else % if luatex or xelatex
  \ifxetex
    \usepackage{mathspec}
    \usepackage{xltxtra,xunicode}
  \else
    \usepackage{fontspec}
  \fi
  \defaultfontfeatures{Mapping=tex-text,Scale=MatchLowercase}
  \newcommand{\euro}{€}
\fi
% use upquote if available, for straight quotes in verbatim environments
\IfFileExists{upquote.sty}{\usepackage{upquote}}{}
% use microtype if available
\IfFileExists{microtype.sty}{\usepackage{microtype}}{}
\ifxetex
  \usepackage[setpagesize=false, % page size defined by xetex
              unicode=false, % unicode breaks when used with xetex
              xetex]{hyperref}
\else
  \usepackage[unicode=true]{hyperref}
\fi
\hypersetup{breaklinks=true,
            bookmarks=true,
            pdfauthor={@cympfh},
            pdftitle={(unlabeled)},
            colorlinks=true,
            citecolor=blue,
            urlcolor=blue,
            linkcolor=magenta,
            pdfborder={0 0 0}}
\urlstyle{same}  % don't use monospace font for urls
\setlength{\parindent}{0pt}
\setlength{\parskip}{6pt plus 2pt minus 1pt}
\setlength{\emergencystretch}{3em}  % prevent overfull lines
\setcounter{secnumdepth}{0}

\title{(unlabeled)}
\author{@cympfh}

\begin{document}
\maketitle

\section{;; レイカの記憶}\label{ux30ecux30a4ux30abux306eux8a18ux61b6}

最初に思ったのは、
私は後ろを見ていたのに気づいたら前を見ていたということだった.
これは要するに、私が寝ていたことを意味する.

その日はなんでもない、つまらない月が上がっているだけで、
その他には何も見えなかった.

目を瞑る前は何も見えない真っ暗だったのに、 あんまり真っ暗だったから、
しょうがなく目を瞑ったのに、 お日様の眩しさで目が覚めた.

自分が月の上にいることに気づくのには、そんなに時間がかからなかったと思う.
だってこの光景を自分は知っているから. まるで三日月のような地球の写真.
これみよがしに写真の下の方には月面がちらりと入っているの.

でもこの認識がニセモノだと気づくのには随分かかったな.
だって何も矛盾は見つけられなかったんだもの.

まず最初に私がしたことといえば、
自分の手が自分の意志通りに動くのを確かめたこと
あと自分の手のひらを見て、
わたしったら手のシワが濃くてなんだかおばあちゃんみたいだって
そう思ったこと.

それから、 それから何を話そうかな.

自分が月にいるって気づいてから
月面の上に立った生物は歴史上何人目だったかの確認をしたわ.
1969年の7月にアポロ11号が月の上に降り立って、
アメリカ人が12人、月の上を歩いたはず.

それが実は嘘だっていう話もあたしの知識の中にはあったけど、
それは無視した. あんまり深い探索はしないことにしてたから.

今の日付は2011年の7月24日.\\ちょっと驚いたのは、
それ以上に正確な時刻を知る術を、あたしはもっていなかったということ.
日付以上に細かい時を知る必要はないと考えたのかしら. ともかく、
それで、私が持ってる知識があんまり古いことがわかったら、
人数を数えるのを諦めたの.

それから、あたしの名前はレイカっていうらしい.
日本人の名前なのは、私を書いた人間が日本人だったからみたい.

\section{(begin}\label{begin}

これはあなたの街にも、もしかしたら当てはまるから、確かめてみるのは面白いと思うのだけれど、
駅のあちら側とこちら側で景色がまるで違っていることはよくある.
きっと線路というものは強く境界線として働くのだと思う.

線路よりこちらにはお店がたくさんあるのに対して、
その向こうには何もない.

私は一度だけ、あの線路の向こう側に行ったことがある.
とても淋しい場所だと思った.
そこでは人は、朝、出かける時と夕方になって家に帰るときくらいしか、
外にいない. だから一日の内ほとんどいつも、外で人を見かけることがない.
そこに一人立っている私は異端者だった.
だって、ただ住宅地があるだけの街に、家もない私が立っているのだから.

「遠子?ねえ」

という妄想をしていると、

「今、遠子に話しかけてんだけど?」

と邪魔をしてくる者、すなわち友達、を再発見した.
彼女たちは友達、間違いなくそう、だと思う. 一緒にご飯だって食べる.
それだって邪魔だと思う.\\曰く、次の英語サボる? だとか
(なんだその疑問)\\曰く、あの先生なんか怖いよね?
(とんでもない)\\本当に、迷惑.\\だって、友達にはいつだって同意してあげるものでしょう?

私達はコンビニで買った菓子パンを食べて、
主に今そこで授業をやってる先生のモノマネで盛り上がった.

\begin{itemize}[<+->]
\itemsep1pt\parskip0pt\parsep0pt
\item
  誰が、私達5人が
\item
  何を、ランチを
\item
  どこで、
\end{itemize}

あ、そうだ. ちょっと語りを変えましょう.
うん、これはとても面白い試みだわ. だって、これは
これはホントに面白いミステリーなんだから.
せっかくの、とびっきりの食材を調理しないでそのまま食べる手なんて、
ないでしょう?

それにあたし、 あの先生のしてくださる授業は痛快で大好きなんだから.
あ、ごめんね、桃香は嫌いだって言ってたけど.

この予備校の中でも11号館は飛び抜けて古くって、
外壁には甲子園球場みたいに蔦が絡んでて、
1106教室は11号館入ってすぐの大きな階段を登った所にある.
11号館の中身はちょっと面白い.
予備校に似つかわしくないくらい、なかなか荘厳で、
二階建てのちょっとしたドームみたいな建物なのだけど、
1階にはまあるい、大きなロビーがあって、 それから、
吹き抜けになってるから2階の廊下からはロビーが丸々見下ろせるの.
ロビーは柱を取り囲むようにソファが置いてあるのだけれど、
柱のところにゴミ箱が置いてあるせいで、
せっかくの美観が台無しになっている.

13時、授業が始まる. 講座の名前は英文例文読解という.
どうせ最初の30分くらいは生徒が遅れて入ってくるので、
教室のドアを開けておくのが通例になっている.
そしてそのためもあってか、最初の30分は雑談が多めだ.
わざと授業の進行が遅らされている配慮にどのくらいの人たちが気づいてるだろう.
(結局、5人が最初の30分間に入ってきた)

開きっぱなしの後方のドアから笑い声が聞こえてきた.
女の子たちの、黄色い声.
黄色というのは、手を叩いて笑う、姦しい下品さを言う.
私は静かに立ち上がり、教室の後方に向かい、 扉から出た.
足元にペットボトルが転がっていた.
そして手すりから見下ろすと、女子4人がソファでだべっているのを見つけた.
私はペットボトルを広い、柱のそばのゴミ箱に向かって投げた.
そのあとペットボトルがゴミ箱に入ったのか、
あるいはどう反射したのか目で確認する間もなく教室に入り、ドアを閉めた.

さて運悪く、二階から降ってきたペットボトルが床に反射し、顔面を直撃した
女の子は、他の子が止める素振りをするのを見ず30秒後に教室の扉を開けて
犯人を探した.
教官は黒板の脇に置いてある椅子に座り説明をしているか、さもなくば雑談をし、
学生はみな静かに座り板書を必死に写す者あれば教官を見てやたら頷いていた.
犯人はすぐに見つかった.
そいつは扉から一番遠いところで座っている人間.つまり、英語の先生だった.

不思議なことに、 他の学生から、先生が教室を出てものを投げたなんていう
目撃情報は得られなかった.

私がひとしきり喋り終えると、他の4人は遠巻きに私を見つめていた.
遠巻きというのは精神的な距離っていう比喩なのだけどね.
一人が無理に笑い、苦笑いと愛想笑いの違いを私は知らない、桃香が明らかに怒ってた.
それを見やるとまた別の一人がちょっと、と言う.

「桃子、ほんと変わってるよね」

さてマクドナルドに入ろうということになって、
あんまりいいタイミングでもないけど、
それでもこれ以上には無さそうな離別のタイミングだと考えて、
私はここで別れることにした.
「今食べちゃうと家で食べれなくなっちゃうから」
くらいの言い訳は考えて言った. 友達だからね.

そうして私は一人になった. 一人で駅のホームに上がり、
一人で電車に乗った.
電車から外の風景を見下ろしていれば、すぐに私の家の最寄り駅に着く.

見下ろすっていうのは、正しく適切な表現で、
この線はここから3駅の間、すこし高いところを走る.
地上は大きな下り坂になっていて、 大抵、人はあんまり歩いていない.
つまらない町だと思う.

一人の少女が歩いてるのを見た. そして私は、
「彼女はどうして一人なのだろう」 という議論を始めた.
「つまり、彼女は私と同じなんだ」 という説が濃厚だった.

\section{(let/cc}\label{letcc}

坂を下る少女が私と同じだと考えた途端、 彼女のことをかわいそうに思えた.

それはつまり、私は私のことをかわいそうなの?
という疑念は切って捨てなければならない、と自分に言い聞かせた.
他人だからこそ、かわいそうなのだ。 自己憐憫はもっとかわいそうだ.
自分をかわいそうと思うのは世界で一番かわいそうな人だけの権利でしょう。
自己憐憫は一種の快楽だ。 全く何も産まない、非生産的な快楽だ。
そんなことは、ちょっと頭を掠ったところで、五秒で止めるべきだわ。
そんな人はわざと自分で自分を貶める。
それでそんな状況を誰かにひけらかしたがる。
凛々しい猫よりも、濡れて泥に汚れたしょぼくれた猫のほうが大切に扱われると信じてる。
そんな姿、見せることを恥だと思わない。 見せられた方はたまらないと思う。
ああ、つまらないことを考えた。 ようするに、ね。
私は可哀想ではないから、もっと下を見るのだ.
言い換えればもっとかわいそうな人がどこかにいるのだ。
例えば、そこにいる、 可哀想な少女は、地面にしゃがみこんで、
そうして、私と彼女との距離は見えないくらいに広がった.
まったく可愛そうだ!

\begin{center}\rule{3in}{0.4pt}\end{center}

坂を降りる右手にお店が一つだけあるけれど、
来年ここを去るまでには一度は行ってみたいとは思うけれど、
女一人で入るには少し憚られる. 他には何もない坂だ. 本当に面白くない.
物語が起こるとしたら、もっと先.
坂がちょうど終わって、高速道路が上を直行する箇所はトンネルになっていて、
歩道は迂曲する.
私が猫を触ることのできるのは、駅にある緑地公園とここだけだ.

人が通りかかるまで
大きな猫も小さな猫も平気で道路の上に寝そべってくつろいでいる.
私が現れるやいなや、大きな猫だけが逃げるのだ.
皆同じ方に逃げるので蜘蛛の子を散らすと書くのは誤りで、
それまで喧嘩をしてたように睨み合ってた二匹が仲良く、
同じ場所に逃げて草場から私を眺めていた光景を、私は見たことがある.
さて子猫は、大人たちが逃げるのを見て、自分も逃げるべきなのかと踵を返そうともするが、
人間たちに対する興味が結局は、自分たちをその場に留める.
逃げ忘れたなんてものじゃない. 腹を見せて媚びてくるのだから.

それで、ね。
信じないかもしれないけれど、猫というものは人間の言葉を解する。
諸君、これは本当のことだ。
でもなければ!\\犬のように忠実でもなく、雀のように空に飼われてもないその小動物が、
野良を保って今もなお、繁栄できているはずがないではないか。
詳細はポール・ギャリコの猫語の教科書を読むとよろしいが、
英語圏の野良猫は英語を読み書きできるし、
江坂に住み着く一匹の野良猫は私に日本語で話しかけてきた.
ただ訛りがひどくって、おまけに私自身が人との会話に不慣れな為に、
上手く聞き取ることはできなかったけれど.
でも確か、食べ物を持ってないか、みたいなことを私に言ったんだ。
何か小魚の、おつまみみたいなおやつでも持ち歩いていれば良かった、
と後悔したのは無理ない。 騙す形になって申し訳ないけれど、
カバンに手を入れ、何か取り出そうというフリだけだけし、 歩み寄った。
子猫はお腹を仰向けにし、まるで背中をコンクリートに擦り付けるみたいに体を揺らした。
猫が精一杯人間に媚びるポーズだ。 上空から電車が通る音がし、
私だけが驚いて、 釣られて猫も驚いた。 教訓は二つ。
猫は電車の音には驚かない。 人間がビクビクしないことである。
それから、何が悪かったのかな。 やっぱり餌を持ち歩いてなかったことかな。

\section{k}\label{k}

私にとって江坂という街はただ細長い帯のような形をしている。
細長くてその真ん中を、軸のように線路が走っていて、
線路によって右と左とに完全に分割されている。
線路といったけれど実際には線路は上空を走っている。
江坂は文字通り坂なのだけれど、それもなかなかに急な下り坂なのだけれど、
線路はできるだけ水平なところを走るべきだ。
そういうわけで、車道があって、その上空を線路が走っている。
さて、気がついたことがある。
ほとんど人通りがないのはいつものことだけれど、
先程から車が一台も走っていない。 電車も、さっき一台だけ
(電車って一台って数えるんだっけ。複数の車両が結ばれたものだから、一本という言い方があったわね)
坂の上の方から来たのが最後だった。
トンネルの内側にはDFSとスプレーで書かれた落書きがいつもどおりあって、
抜けると深い溝があって、その中には一台 (これは一台、よね。一両、かしら)
銀色の自転車が錆びてある。 私にとってのいつもどおりの光景だ。
他に見るべき光景がないので、これ以上観察しても何もわからなかった。
駅が近くになって、ちょっとばかり栄えた街に来たけれど、
例えばスーパーがあって、
自転車の出し入れするおばさんが、私の通行の邪魔をする。
そのスーパーが閉店していた。 例えば私がいつも何か買おうかなと思って、
その品揃えの悪さに結局、何も買うものがなくて出るコンビニ。
やっぱりそのコンビニも閉店していた。
私は閉店するコンビニというのを、二店しか知らない。
これはかなりの確率だと思った。 つまり大変有益な情報なのだ。
だから何度繰り返して書いても怒られないと思う。
今まで24時間営業だと思っていたコンビニが閉まっていた。
コンビニにシャッターはない。
ただあの自動ドアが閉まっていて、私が近づいても開かない。 中は真っ暗で、
私はその日初めて太陽が明るく、今日が雲ひとつない快晴であることを知った、
表面反射というのだっけ、
中が暗くて、外が明るいためにガラスがほとんど鏡のようになって中が伺えない。
私は間近まで近づいて、両手で覆うようにして中を覗きこんだ。
全くの予想通り、店員はいなかった。
このような状況に陥った人は、何を考えてどんな行動をするだろうか。
これに近い状況なら、皆さん体験したことがあるでしょう。
放課後の誰もいなくなった学校に忍び込んだとか、
廃墟同然のビルに入り込んだとか、 そういう感じ。

今はそれがちょっと、街全体と、範囲が大きくなったというだけ。
私は断然、興奮してその状況を楽しんでいた。
私の通う予備校のすぐ近くに、府内でも有数な、大きな公園があるのだけど、
人に会わずに散歩をする方法を、私は熟知している。
噴水の周りにはベビーカーを引いた中年女性やらボール遊びをする子供やらがおり、
おそらく彼ら彼女らは知らない多くの小道がある。
人のいないところには大抵浮浪者がいて、
読者諸君の中に、あるいは身内に、浮浪者がいる人たちには申し訳ないが、
いや、悪く誤解しないで欲しいのだけれど、
そういった俗世間を捨てた人たちは、
私の散歩を脅かす相手にはならなかった。
私にとって彼らの動きを見るのは、野良猫を眺めるのと同じくらいのものなんだ。

なんだかまるで趣味が悪いことを口走っちゃったかしら。
とにかく、この街には誰もいない。最後に見たのは子猫で、
迷惑なおばさんもコンビニ店員も浮浪者もいない。
そんな街を私はワクワクして散歩した。
同様に自動ドアの開かないBOOK-OFFに、
シャッターを降ろしたドラッグストアをじろじろと眺めながら、
天下の往来をふらふら歩くのは楽しいもの。
機会があれば、皆さんも是非試してください。

\begin{center}\rule{3in}{0.4pt}\end{center}

普通に生きていれば、そう何度も聞くことのない音が聞こえた。
一種の恐れを感じさせるけれど、それよりは気持ち良さの方が大きな音。
限界まで耐えて耐えて、一気に崩壊するような、そんな快感。

\begin{quote}
自然は安定を求める。不安定から安定に落ちるときの、その落差が熱だとか音となって放出せらるる。
自然現象というのはいつだって、不安定の状態から安定した状態への遷移だ。
\end{quote}

自分以外の存在を認めた私は、急に不安になったわ。
誰だって、絶対に誰もいないつもりで歩いてた道の向こうから、
誰か大人が来たら踵を返すでしょう? もちろん逃げるでしょうね?
でも少し考えて、 逃げないことにした。
このまま散歩してる場合じゃないって、やっと気づいた。
電車も走ってないのなら、私は帰ることができない。

TODO

\end{document}
