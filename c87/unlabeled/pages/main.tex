\documentclass[]{book}
\usepackage[T1]{fontenc}
\usepackage{lmodern}
\usepackage{amssymb,amsmath}
\usepackage{ifxetex,ifluatex}
\usepackage{fixltx2e} % provides \textsubscript
% use upquote if available, for straight quotes in verbatim environments
\IfFileExists{upquote.sty}{\usepackage{upquote}}{}
\ifnum 0\ifxetex 1\fi\ifluatex 1\fi=0 % if pdftex
  \usepackage[utf8]{inputenc}
\else % if luatex or xelatex
  \usepackage{fontspec}
  \ifxetex
    \usepackage{xltxtra,xunicode}
  \fi
  \defaultfontfeatures{Mapping=tex-text,Scale=MatchLowercase}
  \newcommand{\euro}{€}
\fi
% use microtype if available
\IfFileExists{microtype.sty}{\usepackage{microtype}}{}
\ifxetex
  \usepackage[setpagesize=false, % page size defined by xetex
              unicode=false, % unicode breaks when used with xetex
              xetex]{hyperref}
\else
  \usepackage[unicode=true]{hyperref}
\fi
\hypersetup{breaklinks=true,
            bookmarks=true,
            pdfauthor={@cympfh},
            pdftitle={(unlabeled)},
            colorlinks=true,
            urlcolor=blue,
            linkcolor=magenta,
            pdfborder={0 0 0}}
\urlstyle{same}  % don't use monospace font for urls
\setlength{\parindent}{0pt}
\setlength{\parskip}{6pt plus 2pt minus 1pt}
\setlength{\emergencystretch}{3em}  % prevent overfull lines
\setcounter{secnumdepth}{0}

\title{(unlabeled)}
\author{@cympfh}
\date{}

\begin{document}
\maketitle

\section{レイカの記憶}

最初に思ったのは、
私は後ろを見ていたのに気づいたら前を見ていたということだった。
これは要するに、私が寝ていたことを意味する。

その日はなんでもない、つまらない月が上がっているだけで、
その他には何も見えなかった。

目を瞑る前は何も見えない真っ暗だったのに、 あんまり真っ暗だったから、
しょうがなく目を瞑ったのに、 お日様の眩しさで目が覚めた。

自分が月の上にいることに気づくのには、そんなに時間がかからなかったと思う。
だってこの光景を自分は知っているから。 まるで三日月のような地球の写真。
これみよがしに写真の下の方には月面がちらりと入っているの。

でもこの認識がニセモノだと気づくのには随分かかったな。
だって何も矛盾は見つけられなかったんだもの。

まず最初に私がしたことといえば、
自分の手が自分の意志通りに動くのを確かめたこと
あと自分の手のひらを見て、
わたしったら手のシワが濃くてなんだかおばあちゃんみたいだって
そう思ったこと。

それから、 それから何を話そうかな。

自分が月にいるって気づいてから
月面の上に立った生物は歴史上何人目だったかの確認をしたわ。
1969年の7月にアポロ11号が月の上に降り立って、
アメリカ人が12人、月の上を歩いたはず。

それが実は嘘だっていう話もあたしの知識の中にはあったけど、
それは無視した。 あんまり深い探索はしないことにしてたから。

今の日付は2011年の7月24日。\\ちょっと驚いたのは、
それ以上に正確な時刻を知る術を、あたしはもっていなかったということ。
日付以上に細かい時を知る必要はないと考えたのかしら。 ともかく、
それで、私が持ってる知識があんまり古いことがわかったら、
人数を数えるのを諦めたの。

それから、あたしの名前はレイカっていうらしい。
日本人の名前なのは、私を書いた人間が日本人だったからみたい。

\section{(begin}

これはあなたの街にも、もしかしたら当てはまるから、確かめてみるのは面白いと思うのだけれど、
駅のあちら側とこちら側で景色がまるで違っていることはよくある。
きっと線路というものは強く境界線として働くのだと思う。

線路よりこちらにはお店がたくさんあるのに対して、
その向こうには何もない。

私は一度だけ、あの線路の向こう側に行ったことがある。
とても淋しい場所だと思った。
そこでは人は、朝、出かける時と夕方になって家に帰るときくらいしか、
外にいない。 だから一日の内ほとんどいつも、外で人を見かけることがない。
そこに一人立っている私は異端者だった。
だって、ただ住宅地があるだけの街に、家もない私が立っているのだから。

「遠子?ねえ」

という妄想をしていると、

「今、遠子に話しかけてんだけど?」

と邪魔をしてくる者、すなわち友達、を再発見した。
彼女たちは友達、間違いなくそう、だと思う。 一緒にご飯だって食べる。
それだって邪魔だと思う。\\曰く、次の英語サボる? だとか
(なんだその疑問)\\曰く、あの先生なんか怖いよね?
(とんでもない)\\本当に、迷惑。\\だって、友達にはいつだって同意してあげるものでしょう?

私達はコンビニで買った菓子パンを食べて、
主に今そこで授業をやってる先生のモノマネで盛り上がった。

\begin{itemize}
\itemsep1pt\parskip0pt\parsep0pt
\item
  誰が、私達5人が
\item
  何を、ランチを
\item
  どこで、
\end{itemize}

あ、そうだ。 ちょっと語りを変えましょう。
うん、これはとても面白い試みだわ。 だって、これは
これはホントに面白いミステリーなんだから。
せっかくの、とびっきりの食材を調理しないでそのまま食べる手なんて、
ないでしょう?

それにあたし、 あの先生のしてくださる授業は痛快で大好きなんだから。
あ、ごめんね、桃香は嫌いだって言ってたけど。

この予備校の中でも11号館は飛び抜けて古くって、
外壁には甲子園球場みたいに蔦が絡んでて、
1106教室は11号館入ってすぐの大きな階段を登った所にある。
11号館の中身はちょっと面白い。
予備校に似つかわしくないくらい、なかなか荘厳で、
二階建てのちょっとしたドームみたいな建物なのだけど、
1階にはまあるい、大きなロビーがあって、 それから、
吹き抜けになってるから2階の廊下からはロビーが丸々見下ろせるの。
ロビーは柱を取り囲むようにソファが置いてあるのだけれど、
柱のところにゴミ箱が置いてあるせいで、
せっかくの美観が台無しになっている。

13時、授業が始まる。 講座の名前は英文例文読解という。
どうせ最初の30分くらいは生徒が遅れて入ってくるので、
教室のドアを開けておくのが通例になっている。
そしてそのためもあってか、最初の30分は雑談が多めだ。
わざと授業の進行が遅らされている配慮にどのくらいの人たちが気づいてるだろう。
(結局、5人が最初の30分間に入ってきた)

開きっぱなしの後方のドアから笑い声が聞こえてきた。
女の子たちの、黄色い声。
黄色というのは、手を叩いて笑う、姦しい下品さを言う。
私は静かに立ち上がり、教室の後方に向かい、 扉から出た。
足元にペットボトルが転がっていた。
そして手すりから見下ろすと、女子4人がソファでだべっているのを見つけた。
私はペットボトルを広い、柱のそばのゴミ箱に向かって投げた。
そのあとペットボトルがゴミ箱に入ったのか、
あるいはどう反射したのか目で確認する間もなく教室に入り、ドアを閉めた。

さて運悪く、二階から降ってきたペットボトルが床に反射し、顔面を直撃した
女の子は、他の子が止める素振りをするのを見ず30秒後に教室の扉を開けて
犯人を探した。
教官は黒板の脇に置いてある椅子に座り説明をしているか、さもなくば雑談をし、
学生はみな静かに座り板書を必死に写す者あれば教官を見てやたら頷いていた。
犯人はすぐに見つかった。
そいつは扉から一番遠いところで座っている人間。つまり、英語の先生だった。

不思議なことに、 他の学生から、先生が教室を出てものを投げたなんていう
目撃情報は得られなかった。

私がひとしきり喋り終えると、他の4人は遠巻きに私を見つめていた。
遠巻きというのは精神的な距離っていう比喩なのだけどね。
一人が無理に笑い、苦笑いと愛想笑いの違いを私は知らない、桃香が明らかに怒ってた。
それを見やるとまた別の一人がちょっと、と言う。

「桃子、ほんと変わってるよね」

さてマクドナルドに入ろうということになって、
あんまりいいタイミングでもないけど、
それでもこれ以上には無さそうな離別のタイミングだと考えて、
私はここで別れることにした。
「今食べちゃうと家で食べれなくなっちゃうから」
くらいの言い訳は考えて言った。 友達だからね。

そうして私は一人になった。 一人で駅のホームに上がり、
一人で電車に乗った。
電車から外の風景を見下ろしていれば、すぐに私の家の最寄り駅に着く。

見下ろすっていうのは、正しく適切な表現で、
この線はここから3駅の間、すこし高いところを走る。
地上は大きな下り坂になっていて、 大抵、人はあんまり歩いていない。
つまらない町だと思う。

一人の少女が歩いてるのを見た。 そして私は、
「彼女はどうして一人なのだろう」 という議論を始めた。
「つまり、彼女は私と同じなんだ」 という説が濃厚だった。

\section{(let/cc}

坂を下る少女が私と同じだと考えた途端、 彼女のことをかわいそうに思えた。

それはつまり、私は私のことをかわいそうなの?
という疑念は切って捨てなければならない、と自分に言い聞かせた。
他人だからこそ、かわいそうなのだ、と思う。
自己憐憫はもっとかわいそうだ。
私は可哀想ではないから、もっと下を見るのだ。
可哀想な少女は地面にしゃがみこんで、
そうして私と彼女との距離は見えないくらいに広がった。
まったく可愛そうだ!

\begin{center}\rule{3in}{0.4pt}\end{center}

坂を降りる右手にお店が一つだけあるけれど、
来年ここを去るまでには一度は行ってみたいとは思うけれど、
女一人で入るには少し憚られる。 他には何もない坂だ。 本当に面白くない。
物語が起こるとしたら、もっと先。
坂がちょうど終わって、高速道路が上を直行する箇所はトンネルになっていて、
歩道は迂曲する。
私が猫を触ることのできるのは、駅にある緑地公園とここだけだ。

人が通りかかるまで
大きな猫も小さな猫も平気で道路の上に寝そべってくつろいでいる。
私が現れるやいなや、大きな猫だけが逃げるのだ。
皆同じ方に逃げるので蜘蛛の子を散らすと書くのは誤りで、
それまで喧嘩をしてたように睨み合ってた二匹が仲良く、
同じ場所に逃げて草場から私を眺めていた光景を、私は見たことがある。
さて子猫は、大人たちが逃げるのを見て、自分も逃げるべきなのかと踵を返そうともするが、
人間たちに対する興味が結局は、自分たちをその場に留める。
逃げ忘れたなんてものじゃない。 腹を見せて媚びてくるのだから。

それでだ。
信じないかもしれないけれど、猫というものは人間の言葉を解する。
でもなければ!\\犬のように忠実でもなく、雀のように空に飼われてもないその小動物が、
野良を保って繁栄できたわけがない。
詳細はポール・ギャリコの猫語の教科書に任せるが、
きちんと、英語圏の野良猫は英語を読み書きできるし、
江坂に住み着く野良猫は私に日本語で話しかけてきた。
ただ訛りがひどくって、おまけに私自身が人との会話に不慣れな為に、
上手く聞き取ることはできなかったけれど。

\subsection{2014年 7月 21日 月曜日 21:13:56 JST}

男と女の会話。 男は誰とも付き合いたくないと思っている。

じゃあつまり、お前は俺の腕を好きになったわけだな?

じゃあさ、一千万で売るよ。そしたらもう俺に二度とつきまとうな。

ばか。一千万なんてお前あるわけないだろ。

絶対に貯めてやるという言葉にたじろいだ。
いたって本気で正気であるという目が怖くて、 男も冷静な判断を欠いた。
本当にいつか一千万貯めて、
そうしていつか、自分の腕を切り落としに来る気がした。

\begin{center}\rule{3in}{0.4pt}\end{center}

ねぇ、 両腕なら、ちょっとおまけしてくれる?

両腕は困るな。仕事ができなくなる。

じゃあ、 どっちの腕を売ってくれるのかしら。

男はしばらく考えた。 利き手は右でも、どうせ片腕じゃ何もできない。
それなら腕時計をはめるための左腕は残しておくのが 一番マシな気がした。

\end{document}
