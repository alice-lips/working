
「ねむ……」\\
「おっ、唯ちゃん。眠いん?」\\
「唯ちゃん、ラムシュタイン?」\\
「ネタが分からん……」\\
 今日も今日とて不思議な会話を教室で繰り広げる三人。唯にゆずこ、それと縁。いつも仲が良く、仲が良すぎる。それ故、たまに変な誤解を招く。それでも、この三人の繋がりが消えてなくなることは最後の審判が訪れようときっとないだろう。\\
「ぷよぷよかなー?」\\
「ぷよぷよは4つだろ。コラムスとか」\\
「パズル玉とか?」\\
「唯ちゃんはマネーアイドルエクスチェンジャーだよね!」\\
「落ちゲーとはちょっと違うだろ……。それにもう消える数が関係ない」

 ——そんな三人を遠くの席から窺い見る者たちがいる。いや、それも同じく三人組だ。今回のメインキャストである。\\

「櫟井さんたちってホントに仲良いよね~」\\
「何話してんのかワケワカんないけどな」\\
「桂もそうだけどね」\\
「あ” ?」\\
 この三人組は千穂に桂、それとふみである。先の三人組と同じように仲が良い。この三人もまた、ラグナロクが起ころうとも繋がりが消えてなくなることはないだろう。恐らく。多分。\\
「私たちって櫟井さんたちに例えるなら誰なのかな」\\
「あー。どうだろ」\\
「千穂は日向さんって感じがする」\\
「じゃあ、ふみおちゃんは野々原さんかな」\\
「(残る櫟井があたしか……)」\\
 しばし考えに耽る三人。\\
「よく分からないね~」\\
「よく分からないね」\\
「よく分からんな」\\
 先の三人とこの三人。似ているとは言われれば似ている。しかし、他人は他人であるからにしてそっくりそのままということはない。親と子でさえ違うのだ。\\
「私、お嬢様じゃないし……」\\
「私も野々原さんみたいに頭良くないよ」\\
「(櫟井よりあたしのほうが相川のことが好きだ)」\\
「おかちー?」\\
「……なんでもないっ」\\
 なぜか赤面する桂を千穂は不思議がる。ふみはなぜかニヤニヤとしている。この三人も傍から見れば変な関係なのかもしれない。\\
「んー、じゃあ私たちで櫟井さんたちの真似してみる?」\\
 その千穂からの提案にふみは首を傾げながら答える。\\
「……なぜに?」\\
「ふぇっ!? えと、そのー、……ね! 真似してみると櫟井さんたちのこと分かるかなーって」\\
「分かる分からないってそういう話なのか……?」\\
 といいつつも千穂の提案なので桂が断ることはない。ふみも紙パックのマミーをストローでズゾゾと飲み干し「仕方ねぇな」と言った感じである。\\
「まずはあの三人の特徴を挙げるか」\\
「ん~、櫟井さんはおさげ、とか?」\\
「櫟井さんは意外とおっぱいデッカイ、とか?」\\
「違うわボケ! 変なところの特徴挙げんな! それに……」\\
「それに?」\\
「相川のほうが……」\\
「やぁー、妬けますなぁー。千穂のほうがなんだって?」\\
「うっせ! 死ね! バーカ!」\\
 激昂する桂を見てふみがケラケラと笑う。こんな風に桂はいつもふみのおもちゃである。それを横から申し訳無さそうに笑って見ている千穂という図がこの三人のテンプレだ。\\
「そーいう特徴じゃなくて、性格とか癖だよ」\\
「櫟井さんたちの癖……」\\
「桂のほうが櫟井さんより性格悪いってこと?」\\
「お前、そろそろドツキ回すぞ……?」\\
 フシャー、と猫のように怒る桂にふみはまた腹を抱えて笑う。\\
「ま、まあ。落ち着こ、おかちー? 櫟井さんたちのことでしょ?」\\
「ん。ああ、そうだった。長谷川、なんかある?」\\
「何かって言われると、あー、まあこれ三人の特徴なんだけど。よく擬音で会話してるよね」\\
「あ、なんかそれ分かる気がするー」\\
「なんかさ、ドーン! とかギュオーン! とかぴょーん! とか」\\
「ぴょーん! ってなんだ……」\\
「オノマトペ的な感じ?」\\
「そうそう、そんなん」\\
 そして、三人はまた考えに耽る。\\
「難しいねー」\\
「難しい」\\
「ムズいな」\\
 確かに先の三人組はオノマトペでよく会話しているように思う。しかし、実際真似をしようと思うとなかなかどうして難しい。\\
「ザギンでシースーの後ギロッポンって感じじゃない?」\\
「それのどこがオノマトペなんだよ……」\\
「アハハハ。でも、ちょっとそれっぽく聞こえるよね」\\
「イーケーはホーチーのことキースーって感じだよね」\\
「麻雀用語にみたいになっとる! しかもさり気なく変なこというな!」\\
「アハハハ……また脱線しちゃった」\\
 千穂は困り顔で笑う。その顔を見ると「まあ、いいか」と思えるのが桂であり、そんな桂に漬け込んでいじわる出来ると思っているのがふみなのである。でもどれも愛情表現だ。だからこそ仲が良い。これだけは先の三人組と変わりないことだろう。\\
「櫟井さんだけの特徴としたら何があるだろうね」\\
「櫟井さんの特徴……ツッコミ役とか」\\
「だからそれ特徴か……?」\\
「じゃあ、そこはやっぱりおかちーと一緒だね!」\\
「ん? そうなん? うーん。そう、なんかなぁ」\\
「桂は、どちらかと言うと突っ込むよりかはネコのほうだよね」\\
「ネコ?」\\
「長谷川! 相川の前で変なこと言うなー!」\\
「……ネコ?」\\
「相川なんでもない! なんでもないぞ! ほら、にゃー。にゃー、って!」\\
「にゃ、にゃー……」\\
「桂かわいい~」\\
「後で絶対ぶっ殺す」\\
「私もおかちー可愛いと思うよ?」\\
「うっ……そ、そうか……」\\
 桂の顔がみるみると紅く染まっていく。その姿を見ながらふみも「可愛いやっちゃなー」思うのだった。\\
「野々原さんの特徴はなんだろ」\\
「野々原はー、うーん。頭がいい馬鹿?」\\
「桂、ストレートに表現するね」\\
「しょうがねーだろ!? 特徴なんだからさ!」\\
「千穂だったもっと上手く表現するよ。ね、千穂」\\
「えぇっ!? えっと。野々原さんは場を盛り上げてくれるムードメーカーだよね。いつもニコニコしてて太陽みたい。だから、こっちまで暖かくなるんだよね、野々原さんといると。それに、頭も良いし」\\
「ほら、千穂のほうがよっぽど野々原さんのこと上手く表現してるよ」\\
「んなもん知るか! だいたい、相川だって頭が良いってのオマケ程度に付け加えてるじゃねーか!」\\
「(と言うよりかは……)」\\
 ふみは思う。目線をゆずこの方へ向けて。\\

「唯ちゃん、学校来る途中で目ん玉落としちゃった。1個貸して?」\\
「貸すか! っていうか貸し借り出来んわ!」\\

(あんなやり取り見てたら誰だって頭が良いこと疑うよなぁ……)\\

「おい、長谷川。今、とんでもなく失礼なこと考えてただろ」\\
「別に。乳酸菌のNEWって何のことなんだろって考えてた」\\
「嘘つけ!」\\
「アハハハ……」\\
「しかし、そう考えると。長谷川と野々原ってあんま似てなくないか?」\\
「うーん、どうだろ。でも野々原さんとふみおちゃんってよく気が合うよね」\\
「うん、合う。生まれた国が一緒なのかも」\\
「一緒だよ……ていうかあたしたち皆一緒だよ……」\\
「まあ、なんか野々原さんと私は同じ生き物って感じはするよ」\\
「私たちも同じ生き物だからな……?」\\
「桂、人間だったの!?」\\
「ほーぅ?」\\
 グリグリグリ。桂がふみの頭の端と端をげんこつでこねくり回す。ふみが「ふあぁ~~~」と効いてるの効いてないのかよく分からない脱力とした叫びをあげた。\\
「あー、えーと。そ、そうだ! おかちー。日向さんは?」\\
「日向? 日向は……おっとり?」\\
「そのへんは千穂と似てるよね」\\
「そ、そうかなぁ~」\\
 えへへ、と千穂が恥ずかしそうに微笑む。別に褒められたわけじゃないぞと桂は思ったが、見ていて可愛いから良しと茶々を入れることは無かった。\\
「あとは、なんだろ。若干、天然? が入ってるよな」\\
「千穂も入ってるね」\\
「えぇ!? そ、そんなことないと思うよ!?」\\
 気づいてないないのか……と桂とふみは思った。だが、そこが可愛い。桂は無言で頷く。\\
「でも、千穂と日向さんは違うとこあるよ」\\
「ん?」\\
「おっぱい」\\
「あ” ?」\\
「ぼいん」\\
「いや、言い方の問題じゃねーって……」\\
「ぼとん」\\
「落ちてる! それ落ちる音!」\\
「アハハハ!!」\\
 千穂はどうやらツボにハマったようで腹を抱えて笑っている。ふみに持って行かれたと思ったが、まあ、千穂が笑えばなんでもいいと桂は思った。\\
「ま、やっぱり根本的には似てねーって」\\
「繋がる部分はあるけどね」\\
「十人十色って言うし、そんなものなのかもね」\\
「その、なんつーか。この三人はこの三人でいいんだと思う」\\
「ほほう? 桂、聞こえなかったからもっかい」\\
「うっせー! 鼻の穴に鉛筆ぶっ刺すぞ!」\\
「桂、鼻の穴じゃ鉛筆は削れないよ……」\\
「アハハハ!! もう、おかちーもふみおちゃんも!! ふふ、アハハハハ!」\\

 こうして、他愛のない女子高生たちの時間は過ぎていく。似ているようで、似ていない2つの三人組。\\

 ——それでも。\\

「わー、あいちゃんめっちゃ笑ってる!」\\
「相川さん何かあったの?」\\
「私も気になるぅー!」\\
「長谷川のせいだよ……」\\
「え、私と桂の愛でこの子はこんなにも幸せそうに笑っているんだよ?」\\
「あ” !?」\\
「アハハハハ!!」\\

 みんな違うのに、みんな仲が良いのだ。
