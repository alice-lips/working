\story{~~~~~~~~~~~~~~~~}{@cympfh}

\def\YUZUKO{

    ゆずこ\hspace{3.4mm}}
\def\YUI{

    唯\hspace{11mm}}
\def\YUKARI{

    縁\hspace{11mm}}

\def\PAUSE{\hspace{8mm}・・・}

\YUZUKO 電車、久しぶりに乗るね〜
\YUI あー、そうかもなー
\YUZUKO いつぶりくらいかな
\YUKARI 去年の海?
\YUI いやいや、さすがにもっと乗ってるだろ
\YUZUKO 電車、あんまり乗らんよね〜
\YUKARI あんまりね〜

\PAUSE

\YUZUKO いろんな人がいるね
\YUKARI うん
\YUZUKO なんか、さ
\YUKARI うん?
\YUZUKO いろんな人がいるんだなーって。思うよね?
\YUKARI あー、わかるー
\YUI …。
\YUI あんまじろじろ見るなよ
\YUKARI えー?
\YUI ゆずこ、他の人のことじろじろ見てただろ?
\YUZUKO なにそれ?
\YUKARI ?
\YUZUKO わたしだけを見てろってこと?
\YUI いやそうじゃなくて…
\YUKARI あ、席空いたよー
\YUZUKO 座る?
\YUI …。

\PAUSE

\YUKARI かわいかったねー
\YUZUKO かわいかった!
\YUI そうだな
\YUKARI でもすごいぐずってたねー
\YUZUKO ぐずってた!
\YUI (なんか、ゆずこみたいだった…)
\YUZUKO …?
\YUI なんだよ?
\YUKARI なんかねー
\YUZUKO ん?
\YUKARI 今の赤ちゃん、ゆずちゃんみたいだった
\YUZUKO ほへ?
\YUI あー、うん…
\YUZUKO えー? なにそれー? それ喜んでいいの?
\YUKARI なんかー、しゃべり方とか?
\YUI 喋ってたか?
\YUKARI だーだーって
\YUZUKO それ、バカにされてる気がするんですが…

\PAUSE

\YUZUKO 喃語。 乳児の発する言葉。 言語を獲得する前段階
\YUI さっきの赤ちゃんだ
\YUKARI わんわんとか?
\YUZUKO んーと。幼児語。乳幼児の会話に用いられる言葉
\YUKARI ふーん?
\YUZUKO わんわんとかは幼児語っていうんだって
\YUZUKO 言語の獲得…
\YUI あ、これテレビで見たことある
\YUI ブーバ・キキ効果
\YUI 2つの図形を見せて、
\YUI どっちがブーバで、どっちがキキかを答えさせるという心理実験
\YUKARI ブーバー?
\YUI うーんと、言葉に意味はなくて、音の響きとかから判断するんだって
\YUZUKO ぶ〜ば〜
\YUKARI なんかー、キキの方が痛そうな感じする
\YUZUKO 黒板みたいな?
\YUKARI あー、確かに
\YUI ん? どっちが黒板?

\PAUSE

\YUZUKO E. Mark Gold さん ${}^{[1]}$
\YUKARI 誰?
\YUZUKO 幼児の言語獲得の定式化を試みた人、だってー
\YUZUKO 幼児は親が話す言葉だけから言語を学習する…
\YUKARI うーん、それでー?
\YUZUKO 正しい言葉だけから正しい言語を学習できる
\YUZUKO これを Learning from positive data 、と言うそうです
\YUI それっ当たり前じゃない? 何が難しいんだ?
\YUKARI うーん、よくわからない
\YUZUKO 子供は学んだ言葉を使って新しく作文するかもしれない
\YUZUKO それが正しい言葉であるかどうかを、親の反応から学習する
\YUZUKO この場合は、「正しくない言葉」も学習できるチャンスがある
\YUKARI ふーん?
\YUZUKO あ、こんな例があるよ \footnote{参考文献 [3] の例を改変}

\begin{boxnote}
Q. 次の $A$ は自然数 ($1, 2, 3, \ldots$) の部分集合です.  $A$ はどんな集合ですか?
$$A = \{2, 4, 6, 8, \ldots\}$$
\end{boxnote}

\YUI 偶数、だろ?
\YUKARI うんうん

\begin{boxnote}
    A. $A$ は偶数全体.

    $$A = \mathbb{N}_{2} = \{2, 4, 6, 8, 10, 12, \ldots\}$$
\end{boxnote}

\YUZUKO ざんね〜ん
\YUZUKO とは限りません

\begin{boxnote}
    A. $A$ は偶数または13の倍数からなる集合.

    $$A = \mathbb{N}_{2} \cup \mathbb{N}_{13} = \{2, 4, 6, 8, 10, 12, 13, 14, \ldots\}$$
\end{boxnote}

\YUKARI えー、インチキだよー
\YUZUKO 最後まで聞かないほうが悪いのです
\YUI お? 急になんだ?
\YUZUKO そうじゃなくってー
\YUKARI あー、 「$\dots$」ね
\YUZUKO そう! 「$\dots$」なのです
\YUI 有限の情報だけから決めるのは無理、ってことか?
\YUZUKO 正しくは positive data (または informant) ね
\YUZUKO これは正しい言葉ですよ、っていう
\YUKARI 言葉?
\YUZUKO あ、そう。これは言葉なのです

\begin{boxnote}
    \begin{itemize}
        \item
            一般に (有限とは限らない) 集合で、学習したい対象のことを Concept という.
            例えば一つの言語 (日本語とか英語とか) は Concept の一例.
            例えば自然数の部分集合は Concept の一例.
        \item
            ある要素が学習したい Concept に属するという情報を positive data (informant) と呼ぶ.
            例えば親が話す言葉は positive data.
            例えば自然数の部分集合 $A$ について $x \in A$ は一つの positive data.
    \end{itemize}
\end{boxnote}

\YUI ってことは集合 $A$ が言語ってことか…?
\YUKARI 自然数が言葉でー?

\PAUSE

\YUI でも、そんなの無理だろ?
\YUZUKO ?
\YUI 有限個の情報、あ、informant か? しかくれなかったんだから
\YUI そこから、そんな、「または13の倍数」なんて分かるわけないだろ?
\YUKARI うんうん
\YUZUKO そこで、「極限同定」という発想が生まれるわけです
\YUI ほーん?

\begin{boxnote}
    次のような学習の仕方を「極限同定」と呼びます.
    \begin{itemize}
        \setlength{\itemsep}{-1mm}
        \item
            ある Concept をこれから学ぼうとする学習者がいます.
        \item
            学習者は一つの positive data (informant) を得ます. \footnote{この情報の得方を正提示 (positive presentation) と言いますが、厳密にはもっと強い制約があります. それは Concept の任意の要素はいつかは必ず提示される、というものです.}
        \item
            学習者はそれまでに得た positive data (informant) から、一つ、考えつく Concept を推論します.
        \item
            上の、positive data を受け取って一つの Concept を推論することを一つのステップとして、これを繰り返します
    \end{itemize}

    更に次のようなとき、極限同定が「成功した」と言います.
    \begin{itemize}
        \setlength{\itemsep}{-1mm}
        \item
            あるステップで真の Concept を推論し、かつ、
        \item
            以降のステップでは、常にその Concept を推論する.
    \end{itemize}
\end{boxnote}

\YUKARI じゃあさっきの例で言うとー
\YUI うん
\YUKARI $2, 4, 6, 8, \ldots$ ってのが正提示でー
\YUI うんうん
\YUKARI 数字一つを受け取るごとに、
\YUI positive data な
\YUKARI 集合を答えるのが、Concept の推論、ってことー?
\YUZUKO そうそう
\YUZUKO ちなみに、推論した Concept の列を「推論列 (guessing sequence) 」っていうよ

\begin{boxnote}
    正提示 $2, 4, 6, 8, \ldots$ に対して、推論列 $A_1, A_2, A_3, A_4, \ldots$.
\[
    \begin{array}{ccccc}
        2, & 4, & 6, & 8, & \ldots \\
        \downarrow & \downarrow & \downarrow & \downarrow & \\
        A_1, & A_2, & A_3, & A_4, & \ldots
    \end{array}
\]
\end{boxnote}

\YUZUKO あ、これは小さい順に数字を並べてるけど、別にそれに意味はなくて
\YUZUKO 順序はどうでもいいし、それに重複してもいいんだって
\YUKARI $2, 10, 2, 10, \ldots$ みたいに?
\YUZUKO うん。でもこの注釈があって
\YUZUKO いつかは 13 の倍数も出現しないといけないわけ
\YUI 時刻 $t$ の推論を $A_t$ って書いたの?
\YUZUKO そうっす
\YUI じゃあさっきの「極限同定の成功」ってのは
$$\lim_{t \rightarrow \infty} A_t = A$$
\YUI って書いていいのか?
\YUZUKO うーん、いいんじゃないかな
\YUZUKO 文字通り形式化すると $\exists T, \forall t > T, A_t = A$ だけどね
\YUZUKO 「極限同定が成功」したとき、Concept を学習したって見做すらしいのさ

\PAUSE

\YUKARI それで?
\YUZUKO うん?
\YUKARI これで、人間の言葉が学習できるの?
\YUZUKO うーん、、、どうなんだろう?
\YUKARI えー
\YUI どうなんすか? ゆずこさーん
\YUZUKO むーん。。。

\PAUSE

\YUZUKO あ!
\YUI なになに
\YUZUKO "文脈自由文法の学習は不可能である" ${}^{[1]}$ \footnote{英語は文脈自由文法であると言われる. いや、少なくとも反証はされていない.}
\YUKARI えーなんでー?
\YUI 推論の具体的な方法も決めてないのに
\YUZUKO TABLE 1 に書いてるのだとー…
\YUI なになに?
\YUZUKO 文脈自由文法も文脈依存文法も正規言語も、正提示からの極限同定は不可能
\YUKARI だからなんでー?
\YUZUKO うーん、詳しい記述はどこにあるんだろう

\PAUSE

\YUZUKO あ、あったあった
\YUZUKO "super-finite class of languages"
\YUKARI クラス?
\YUZUKO うん、言語族、のことかな?
\YUI そんなの今まで話に出てこなかったぞ
\YUZUKO ごめんごめん。説明がめんどそうだったから…
\YUZUKO 「自然数の部分集合」とか「自然言語」みたいに、推論するConcept を中から選べるための枠組みみたい
\YUKARI え? じゃあ、推論って、選択のこと?
\YUI なんか急に問題が簡単に見えてきたなあ

\begin{boxnote}
    列挙による推論.

    Cencept のクラスが添字つき集合 \footnote{集合の要素をもれなく列として並べられるということ} の場合、列挙による推論がありえる. この方法ではまず Concept を列に並べる.
    $$\mathcal{C} = \{ C_1, C_2, C_3, \ldots \}$$
    次のような方法で推論を行う.
    \begin{itemize}
        \setlength{\itemsep}{0mm}
        \item
            添字 $i=1$ を持っておく.
        \item
            positive data (informant) を受け取る.
        \item
            それまでに受け取った positive data と $C_i$ とが矛盾するか調べる.
        \item
            矛盾するなら、添字 $i$ を 1 増やす.
            矛盾しないなら何もしない.
        \item
            推論として $C_i$ を推論する.
    \end{itemize}
\end{boxnote}

\YUKARI さっきの自然数のは? これで解ける?
\YUI いやいや、無理だろ
\YUKARI どうして?
\YUI 自然数の部分集合全体は可算じゃないから
\YUKARI そっかー
\YUZUKO もっと単純なやつじゃないと、これはダメみたいだね
\YUKARI なんならいいんだろう
\YUI 「ある数の倍数」とかじゃないか?
\YUKARI それなら可算だー

\begin{boxnote}
    自然数 $n$ の倍数全体を $\mathbb{Z}_n$ と書く.
    Concept class, $\mathcal{C} = \{ \mathbb{Z}_n : n \in \mathbb{N} \}$ は列挙による方法で正提示から極限同定可能?
\end{boxnote}

\YUZUKO うーん…
\YUZUKO いや、おかしいよ!
\YUKARI えー、なんでー?
\YUZUKO だってほら、4の倍数だとするじゃん?
\YUZUKO 4, 8, 16, \ldots ってきて、
\YUZUKO ほら、2 の倍数の可能性を捨てきれないわけじゃん
\YUI あっ
\YUZUKO でしょ?
\YUI 添字の順序を工夫すれば… うーん… わからんなあ…
\YUKARI どゆこと?
\YUI ほら、$\mathbb{Z}_1, \mathbb{Z}_2$ って順に見てくとするだろ?
\YUI そしたら、正提示はずっと2の倍数だから、添字が 2 で停まるわけよ
\YUKARI あーそっかー
\YUI $\mathbb{Z}_4$ を $\mathbb{Z}_2$ の前に持ってくればいいんだけど、そしたらキリがないし
\YUKARI あーでも、ちょっとズルをしたら上手くいくかも
\YUI どんな?
\YUKARI 最初のpositive data の数だけ先に見ちゃってー、
\YUKARI 例えばそれが $m$ だったら、
$$\mathbb{Z}_{m}, \mathbb{Z}_{m-1}, \mathbb{Z}_{m-2}, \ldots, \mathbb{Z}_{2}, \mathbb{Z}_{1}$$
\YUKARI って並べるの
\YUI あー、それなら上手くいくなー
\YUI 上限を決めて、降りるように並べるわけだな
\YUZUKO ていうか
\YUZUKO これって、提示される数の最小値を取ればいいだけだね、これ \footnote{極小言語 (minimal language; MINL) 戦略の自然数バージョンです}
\YUKARI あー、確かに
\YUI 整数だから、絶対値は取らないとだけどな


\begin{boxnote}
    自然数 $n$ の倍数全体を $\mathbb{Z}_n$ と書く.
    Concept class, $\mathcal{C} = \{ \mathbb{Z}_n : n \in \mathbb{N} \}$ は次のような方法によって正提示から極限同定可能.

    \begin{itemize}
        \setlength{\itemsep}{0mm}
        \item 数 $m = \infty$ を持つ
        \item positive data (informant) $m_t$ を受け取る
        \item $m$ を $m$ と $|m_t|$ の最小値とする ($m \leftarrow \min \{ m, |m_t| \}$) \footnote{$|\cdot|$ は絶対値}
        \item $Z_m$ を推論
    \end{itemize}
\end{boxnote}

\YUKARI でもなんか、
\YUKARI 言葉っぽくないね?
\YUI あ、ていうか、文脈自由文法は? "super-finite" は?
\YUZUKO あ、そうだったそうだった。忘れてた

\PAUSE

\YUI 正提示から学習できる言語。パターン言語 ${}^{[2]}$
\YUZUKO お?
\YUI ンンン?? $\Sigma A$ …?
\YUZUKO どれどれ
\YUI んー。パターン言語の定義らしいんだけど…
\YUZUKO ああ、たぶんこれ、Kleene閉包だよ
\YUI こんなヘンな記法が…

\begin{boxnote}
    非形式的にパターン言語を説明します.
    パターン言語とはあるパターンによって「説明される」言語です.
    そしてここで言うパターンとは、要は、「空欄のある文」です.
    
    \begin{itemize}
        \item 例. ``世界が \fbox{  } でありますように''
    \end{itemize}

    また空欄には「名前」をつけることができます.

    \begin{itemize}
        \item 例. ``\fbox{ (x) } が \fbox{ (x) } を \fbox{ (y) } てました''
    \end{itemize}

    そして、このパターンが「説明する」言語とは、空欄を自由に埋めて出来る文からなる集合のことです.
    ただし、同じ名前の空欄には同じものを埋めます.

    \begin{itemize}
        \setlength{\itemsep}{0mm}
        \item ``私が私を見つめてました''
        \item ``深淵が深淵を覗いてました''
        \item 等々
    \end{itemize}
\end{boxnote}

\YUKARI あー、なんか、言語って感じ
\YUI うーん、そうかー?
\YUI 同じ名前の空欄って、同じものが何度も出現するってことだろ?
\YUI あんまり自然言語にそういうのって出てこないような
\YUZUKO 同じ名前の空欄が出てこないものは、正則パターン言語、と言うそうです
\YUKARI あ、これってもしかして、深さが 1 しかない文脈自由文法?
\YUI あー、なるほど
\YUI ん?

\subsection*{参考文献}

本物語は次の参考文献をヒントに創作しました.

\begin{enumerate}
    \item[$\lbrack 1 \rbrack$] E. Mark Gold: ``Language Identification in the Limit'', in \emph{Information and Control 10 (1967)}
    \item[$\lbrack 2 \rbrack$] Angluin: ``Positive Inference of Formal Languages from Positive Data'', in \emph{Information and Control 45 (1980)}
    \item[$\lbrack 3 \rbrack$] {Hiroki Arimura, Takeshi Shinohara and Setsuko Otsuki}: ``{Finding Minimal Generalizations for Unions of Pattern Languages and Its Application to Inductive Inference from Positive Data}'', in \emph{In Proc. the 11th STACS, LNCS 775 (1994)}
    \item[$\lbrack 4 \rbrack$] {\rm @cympfh}: ``言語の極限同定みたいな話'',\\
        \url{http://cympfh.cc/study/language-identification/history/it.pdf}
\end{enumerate}
