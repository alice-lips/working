\story{~~~~~~~~~~~~~~~~}{@cympfh}

\def\YUZUKO{

    ゆずこ\hspace{4.4mm}}
\def\YUI{

    唯\hspace{12mm}}
\def\YUKARI{

    縁\hspace{12mm}}

\def\PAUSE{\hspace{8mm}・・・}

\YUZUKO 電車、久しぶりに乗るね〜
\YUI あー、そうかもなー
\YUKARI ねー
\YUZUKO いつぶりくらいかな
\YUI あー、いつだろうなあ
\YUZUKO 去年の海以来?
\YUI いやいや、さすがにもっと乗ってるだろ

\PAUSE

\YUZUKO いろんな人がいるね
\YUKARI うん
\YUZUKO なんか、さ
\YUKARI うん?
\YUZUKO いろんな人がいるんだなーって
\YUKARI あー、わかるー
\YUI …。
\YUI あんまじろじろ見るなよ
\YUKARI えー?
\YUI ゆずこ、他の人のことじろじろ見てただろ?
\YUZUKO なにそれ?
\YUKARI ?
\YUZUKO わたしだけを見てろってこと?
\YUI いやそうじゃなくて…
\YUKARI あ、席空いたよー
\YUZUKO 座る?
\YUI …。

\PAUSE

\YUKARI かわいかったねー
\YUZUKO かわいかった!
\YUI そうだな
\YUKARI でもすごいぐずってたねー
\YUZUKO ぐずってた!
\YUI (なんか、ゆずこみたいだった…)
\YUZUKO (じー)
\YUI なんだよ?
\YUKARI なんかねー
\YUZUKO ん?
\YUKARI 今の赤ちゃん、ゆずちゃんみたいだった
\YUZUKO へ?
\YUI うん…。
\YUZUKO えー? なにそれー? それ喜んでいいの?
\YUKARI なんかー、しゃべり方とか?
\YUI 喋ってたか?
\YUKARI だーだーって
\YUZUKO それ、バカにされてる気がするんですが…

\PAUSE

\YUZUKO 喃語。 乳児の発する言葉。 言語を獲得する前段階
\YUI さっきの赤ちゃんだ
\YUKARI わんわんとか?
\YUZUKO んーと。幼児語。乳幼児の会話に用いられる言葉
\YUKARI ふーん?
\YUZUKO わんわんとかは幼児語っていうんだって
\YUZUKO 幼児の言語獲得…
\YUI あ、これ面白い
\YUI ブーバ・キキ効果
\YUI 2つの図形を見せて、どっちがブーバでどっちがキキかを答えさせるっていう有名な心理実験… だってさ
\YUKARI ブーバー?
\YUI うーんと、言葉に意味はなくて、音の響きとかから判断するんだって
\YUZUKO ぶ〜ば〜
\YUKARI あー、キキの方が痛そう
\YUZUKO 黒板みたいな?
\YUKARI あー、確かに
\YUI ん? どっちが黒板?

\PAUSE

\YUZUKO E. Mark Gold さん \footnote{参考文献 [1]}
\YUKARI 誰?
\YUZUKO 幼児の言語獲得の定式化を試みた人、だってー
\YUZUKO 幼児は親が話す言葉だけから言語を学習する…
\YUKARI うーん、それでー?
\YUZUKO 正しい言葉だけから正しい言語を学習できる
\YUZUKO これを Learning from positive data 、と言うそうです
\YUI それっ当たり前じゃない? 何が難しいんだ?
\YUKARI うーん、よくわからない
\YUZUKO 子供は学んだ言葉を使って新しく作文するかもしれない
\YUZUKO それが正しい言葉であるかどうかを、親の反応から学習する
\YUZUKO この場合は、「正しくない言葉」も学習できるチャンスがある
\YUKARI ふーん?
\YUZUKO あ、こんな例があるよ \footnote{参考文献 [3] の例を改変}

\begin{boxnote}
Q. 次の $A$ は自然数 ($1, 2, 3, \ldots$) の部分集合です.  $A$ はどんな集合ですか?
$$A = \{2, 4, 6, 8, \ldots\}$$
\end{boxnote}

\YUI 偶数、だろ?
\YUKARI うんうん

\begin{boxnote}
    A. $A$ は偶数全体.

    $$A = \mathbb{N}_{2} = \{2, 4, 6, 8, 10, 12, \ldots\}$$
\end{boxnote}

\YUZUKO ざんね〜ん
\YUZUKO とは限りません

\begin{boxnote}
    A. 答えは偶数または13の倍数からなる集合でした.

    $$A = \mathbb{N}_{2} \cup \mathbb{N}_{13} = \{2, 4, 6, 8, 10, 12, 13, 14, \ldots\}$$
\end{boxnote}

\YUKARI えー、インチキだよー
\YUZUKO 最後まで聞かないほうが悪いのです
\YUI お? 急になんだ?
\YUZUKO そうじゃなくってー
\YUKARI あー、 「$\dots$」ね
\YUZUKO そう! 「$\dots$」なのです
\YUI 有限の情報だけから決めるのは無理、ってことか?
\YUZUKO 正しくは positive data (または informant) ね
\YUZUKO これは正しい言葉ですよ、っていう
\YUKARI 言葉?
\YUZUKO あ、そう。これは言葉なのです

\begin{boxnote}
    \begin{itemize}
        \item
            一般に (有限とは限らない) 集合で、学習したい対象のことを Concept という.
            例えば一つの言語 (日本語とか英語とか) は Concept の一例.
            例えば自然数の部分集合は Concept の一例.
        \item
            ある要素が学習したい Concept に属するという情報を positive data (informant) と呼ぶ.
            例えば親が話す言葉は positive data.
            例えば自然数の部分集合 $A$ について $x \in A$ は一つの positive data.
    \end{itemize}
\end{boxnote}

\YUI ってことは集合 $A$ が言語ってことか…?
\YUKARI 自然数が言葉でー?

\PAUSE

\YUI でも、そんなの無理だろ?
\YUZUKO ?
\YUI 有限個の情報、あ、informant か? しかくれなかったんだから
\YUI そこから、そんな、「または13の倍数」なんて分かるわけないだろ?
\YUKARI うんうん
\YUZUKO そこで、「極限同定」という発想が生まれるわけです
\YUI ほーん?

\begin{boxnote}
    次のような状況を、極限同定、と呼びます.
    \begin{itemize}
        \item
            ある Concept をこれから学ぼうとする学習者がいます
        \item
            学習者は一つの positive data (informant) を得ます
        \item
            学習者はそれまでに得た positive data (informant) から、一つ、考えつく Concept を推論します
    \end{itemize}
\end{boxnote}


\subsection*{参考文献}

本物語は次の参考文献をヒントに創作しました.

\begin{enumerate}
    \item[$\lbrack 1 \rbrack$] E. M. Gold: ``Language Identification in the Limit'', in \emph{Information and Control 10 (1967)}
    \item[$\lbrack 2 \rbrack$] Angluin: ``Positive Inference of Formal Languages from Positive Data'', in \emph{Information and Control 45 (1980)}
    \item[$\lbrack 3 \rbrack$] {Hiroki Arimura, Takeshi Shinohara and Setsuko Otsuki}: ``{Finding Minimal Generalizations for Unions of Pattern Languages and Its Application to Inductive Inference from Positive Data}'', in \emph{In Proc. the 11th STACS, LNCS 775 (1994)}
\end{enumerate}
