昨日の帰りの電車のこと.
じろじろと私の顔を見てくる男がいた.\\
…あのぅ.
\\
無視されてしまった.
いや、単に気付いてくれてないのかな.
今度はタイミングよく、丁度顔があった瞬間に.\\
あの、何か?\\
え?\\
顔に何かついてますか?\\
いえ、別に。\\
そうですよね。\\
さっき鏡で二回も確認しました.
顔には何もついていません.
男は、
なおも訝しげに私の顔をちらちらと見てくる.\\
何でしょう?\\
尋ねても返って向こうが厄介な人に絡まれた被害者のような顔で\\
いえ、何も\\
という言い草で、すぐさま携帯電話とにらめっこするのを再開した.

それでこんな予想をした.
自分が、この男の携帯を覗き込んでいると勘違いしているのではないか.
しかし、
元はと言えば、そちらが私の方を、無理な角度の首を曲げてまでじろりと見てきて、
そうしたら誰だって、
反射的に見返してしまうものでしょう?
男は、携帯電話の画面に左手で陰を作って見えないようにしてきた.
しかし自分の位置からはそれでも見えるのだ.
やはり男はこちらを何度もちらりと見てくる.
まさか窓の外を見てるふりをしながら、
その反射を使って覗いているとでも思っているのか.
貴方が何度こちらを振り向こうが、私は本当にただ窓の外をただ眺めてるだけなのに.

一体この男は何をそんなにこっそりとすることがあるのだろうか.
やましい事はこんなに人がいる場所でするべきではない.
何、ちょっと見てみれば、携帯のメモ機能で詩を書いていた.\\
面白い詩ですね。\\
あなたの狭い、閉じこもった世界がよく分かります。

\vspace*{4mm}
ワブルベース
\vspace*{4mm}

家に帰ると、
これは別の男が、
しかし本質的にはさっきの男と変わらない男が、
夕飯を机に並べて、テレビを見て待っていた.\\
最近ビジネスを始めたんだ。\\
ふぅん。やっと、って感じね。\\
随分長かったわね、無職期間。\\
まぁそう言うな。\\
何もしてなかった訳じゃないんだ。\\
つまり、新しいビジネスを模索することをしていたんだよ。\\
面白くもないことを言う.\\
それで、何をするの?\\
既存であり、同時に全く新しい仕事だよ。\\
何それ。\\
あんまり変なコトしちゃ嫌よ。\\

これは新しい分野なんだ.
奇怪に思われたって仕方がないや.
しかし、男は彼女にあまり心を開いていなかったので、
自分の発明したビジネスを彼女につい語ってしまったことを後悔し、
またこの事については慎重になろうと思った.

一度だけ実践してみたんだよ。\\
それ程、一度に大金が入るわけじゃないが、\\
一人で出来る分、割はいいかな。

まあ、言ってしまえば、物を安く仕入れて、定価より少し安い値段で売るんだ。\\
もっともこれだけ聞いても、単なる商売の理念そのまんまじゃないか、って思うだろ?\\
そりゃそうね。\\
どこが新しいっていうの?\\
仕入先さ。\\
ちょっと言えないけど、タダ同然で品物が手に入るのさ。\\

自分から言い出したくせに、
急に言葉を濁し始めたので、私もそれ以上詮索しないことにしました.
どうせろくな事じゃないでしょう.
彼が昔何をやっていたのか知りません.
まともな職業に一度でも就いたことがあるのかどうか.
家事のことだけやって、お金のことなんて考えてくれない方が返って安心します.
そうすればずっと一緒にいても良いのに.

血液型占い、というのがあったそうですね.
彼は数年前に流行っていたと言います.
テレビのニュース番組の最後なんかには、
誕生月占いだとか、血液型占いのコーナーがあるものと決まっていたそうです.
私はその頃、勉強ばかりで新聞は読んでもテレビは観る暇がなかったのですが、
観たことのない私にはまだ信じられません.
だって何の根拠もないインチキなんでしょう?
と聞くと、それはそれで大衆に持て囃されていたんだそうです.
本当なのでしょうか.

私の勤め先は病院です.
病院の匂いは好きですが、そこにいる人間のことを考えると
(患者のことではなくて、偉そうにしている医者のことを考えると)
息がつまります.
よくのんびりしていると言われます.
本当はこんなではいけないのです.
輸血のための血液型を調べる仕事を頼まれたのです.
日本人はA型が多いそうなので、
取り敢えずA型で手を打っときましょう、と冗談を言うと本気で叱られてしまいました.
大人にこんなに叱られるなんて高校生以来だから非常にショックを受けたのに、
叱った方はと言うと、何でもない風にするのでなんだか腹が立って.
わざと違った血液型を言ってやろうかしら.

