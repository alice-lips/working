\page{
  はじめまして!  でよろしいですか。
}{
  いいと思いますよ!
}
\page{
  ちなみに『Scheme手習い』は読んだことはありますか。
}{
  \v{\#f} です。
}
\page{
  え、『Scheme手習い』を読んだことがないのですか。
}{
  はい
}
\page{
  それは困りましたね。
}{
  …
}
\page{
  あ、『Scheme修行』なら読んだことがあるとか。
}{
  ええっと…
}
\page{
  線形代数は知っていますか。
}{
  はい、線形代数は親友です。
}
\page{
  状態 $\ket{\varphi}$
  とはなんですか。
}{
  qビット
}
\page{
  $\ket{\varphi} = \alpha \ket{0} + \beta \ket{1}$
}{
  $\alpha, \beta \in \mathbb{R}$ として
  $\alpha^2 + \beta^2 = 1$
}
\page{
  $\alpha, \beta \in \mathbb{C}$
  とするのはまずいですか。
}{
  問題ありません。
  $\alpha^* \alpha + \beta^* \beta = 1$
}
\page{
  $\alpha \in \mathbb{C}$ について
  $\alpha^*$ とはなんですか。
}{
  複素共役です。\\
  $\alpha = a + bi$ ($a,b\in \mathbb{R}$) とおくと
  $\alpha^* = a - bi$
}
\page{
  観測を知っていますか。
}{
  ええ…たぶん。
}
\page{
  観測はqビットに対する正当な操作です。
}{
  なんと。
}
\page{
  $\alpha \ket{0} + \beta \ket{1}$
  を観測するとどうなりますか。
}{
  $\ket{0}$ もしくは $\ket{1}$ のどちらかを観測できます。
}
\page{
  どちらですか?
}{
  $\alpha^2$ の確率で $\ket{0}$ を観測します。
  同様に、
  $\beta^2$ の確率で $\ket{1}$ を観測します。
}
\page{
  $\alpha, \beta \in \mathbb{C}$ かもしれません。
}{
  $\alpha^2$ を $\alpha^* \alpha$、もしくは
   $|\alpha|^2$ と読み替えてください!
}
\page{
  我々はSchemerです。
  $\alpha \ket{0} + \beta \ket{1}$
  を表現してください。
}{
  \v{'(alpha beta)}
}
\page{
  単に並べれば良いのなら、列ベクトル
  $\binom{\alpha}{\beta}$
  でもよさそうですね!
}{
  問題なさそうです。
}
\page{
  ちょうど
  50\% の確率で $\ket{0}$ を観測
  できるような
  qビット
  を想像できますか。
}{
  はい。
  $\frac{1}{\sqrt{2}} \ket{0} + \frac{1}{\sqrt{2}} \ket{1}$
}

\hline
\begin{minipage}{.01\hsize}
  ~
\end{minipage}
\begin{minipage}{.39\hsize}
  先ほどの表現を使いましょう。
\end{minipage}
\begin{minipage}{.10\hsize}
  ~
\end{minipage}
\begin{minipage}{.39\hsize}
  \begin{verbatim}
(define /sqrt2 (/ (sqrt 2)))
(list /sqrt2 /sqrt2)
  \end{verbatim}
\end{minipage}

\hline
\begin{minipage}{.01\hsize}
  ~
\end{minipage}
\begin{minipage}{.39\hsize}
  それだけですか。
\end{minipage}
\begin{minipage}{.10\hsize}
  ~
\end{minipage}
\begin{minipage}{.39\hsize}
  そんなことありません!
  係数はマイナスにしても問題ないので、
  実数に限れば
  \begin{verbatim}
(list /sqrt2 /sqrt2)
(list /sqrt2 (- /sqrt2))
(list (- /sqrt2) /sqrt2)
(list (- /sqrt2) (- /sqrt2))
\end{verbatim}
の4つがあります。
\end{minipage}

\page{
  qビットは観測したら2種類だけが見られましたね。
}{
  $\ket{0}$
  と
  $\ket{1}$
  です。
}
\page{
  もっと多くの種類を観測できたら便利じゃありませんか。
}{
  qビットを並べればよろしいでしょう。
}
\page{
  やってみてもらえますか。
}{
  2つ並べれば
  $\ket{00}$
  $\ket{01}$
  $\ket{10}$
  $\ket{11}$
  の4つになります.
}
\page{
  2qビットの状態はどうなりますか.
}{
  $$\alpha_{00} \ket{00}
   + \alpha_{01} \ket{01}
   + \alpha_{10} \ket{10}
   + \alpha_{11} \ket{11}$$
   ただし
  $\sum |\alpha_{ij}|^2 = 1$
}
\page{
  $\ket{00}$
  $\ket{01}$
  $\ket{10}$
  $\ket{11}$
  の替りに
  $\ket{0}$
  $\ket{1}$
  $\ket{2}$
  $\ket{3}$
  と書いたほうが読みやすいですね。
}{
  ただのエイリアス(別名)ですけどね。
}
\page{
  一般に多qビットが取る状態を記述できますか。
}{
  $m$qビットだとして:\\
  $\sum_{i=0}^{2^m - 1} \alpha_i \ket{i}$
}
\page{
  $\sum_{i=0}^{n - 1} \alpha_i \ket{i}$
}{
  それでいいなら、いいですけど。
}
\page{
  $( \alpha_0, \alpha_1, \ldots, \alpha_{n-1} )^\top$
}{
  係数を並べただけですね。
}
\page{
  これは列ベクトルです。
}{
  $(~)^\top$ はそういうリテラルです。
}
\page{
  これを観測してみてください。
}{
  確率 $|\alpha_i|^2$ で $\ket{i}$ が観測できました。
}
\page{
  観測はそれだけですか。
}{
  とんでもありません!
}
\page{
  もう一度、先ほどの多qビットを観測してください。
}{
  $\ket{i}$ が観測できました。
}
\page{
  その $i$ はもっと前に出てきた $i$ と同じものですか。
}{
  ええ、これからもずっと
  $\ket{i}$ が観測されます。
}

\page{観測という操作を書いてみましょう}{名前は何がいいですか。}

\hline
\begin{minipage}{.01\hsize}
  ~
\end{minipage}
\begin{minipage}{.39\hsize}
  \v{observe} 
  でよいでしょう。
\end{minipage}
\begin{minipage}{.10\hsize}
  ~
\end{minipage}
\begin{minipage}{.39\hsize}
  \begin{verbatim}
(use srfi-1)
(use srfi-27)
(define (observe q)
  (let ((i (random-integer (length q))))
    (map (lambda (j) (if (= j i) 1 0))
      (iota (length q)))))
  \end{verbatim}
\end{minipage}

\page{
  使っている処理系はなんですか。
}{
  \v{Gauche scheme shell, version 0.9.4 [utf-8,pthreads], x86\_64-unknown-linux-gnu}
}

\hline
\begin{minipage}{.01\hsize}
  ~
\end{minipage}
\begin{minipage}{.39\hsize}
  \begin{verbatim}
(define phi
  (list /sqrt2 /sqrt2))
  \end{verbatim}
\end{minipage}
\begin{minipage}{.10\hsize}
  ~
\end{minipage}
\begin{minipage}{.39\hsize}
  \begin{verbatim}
gosh> phi
(0.7071067811865475 0.7071067811865475)
gosh> (set! phi (observe phi))
(0 1)
  \end{verbatim}
\end{minipage}

\page{
  \v{set!} を含めてマクロにしてしまっても良いかもしれませんね。
}{
  それはやめておきましょう。
}

\page{
  \v{'(a b)} に左から
  $\left(
    \begin{array}{cc}
      0 & 1 \\
      1 & 0 \\
    \end{array}
  \right)$
を掛けてください。
}{
  $\left(
    \begin{array}{cc}
      0 & 1 \\
      1 & 0 \\
    \end{array}
  \right)
  \left(
    \begin{array}{c}
      a \\ b
    \end{array}
  \right)
  =
  \left(
    \begin{array}{c}
      b \\ a
    \end{array}
  \right)$
}

\page{どうなりましたか。}
{$\ket{0}$ と $\ket{1}$ を観測できる確率が入れ替わりました。}

\page{ところで、$\ket{0}$ を捕まえることができますか。}
{難しいでしょう}


\hline
\begin{minipage}{.01\hsize}
  ~
\end{minipage}
\begin{minipage}{.39\hsize}
  では、qビットの発生源を用意します。
  \begin{verbatim}
(define (random-q)
  (let ((a (- (random-real) 0.5))
        (b (- (random-real) 0.5)))
    (let ((Z (sqrt (+ (* a a)
                      (* b b)))))
      (list (/ a Z) (/ b Z)))))
  \end{verbatim}
\end{minipage}
\begin{minipage}{.10\hsize}
  ~
\end{minipage}
\begin{minipage}{.39\hsize}
  \begin{verbatim}
  gosh> (observe (random-q))
  (1 0)
  gosh> (observe (random-q))
  (1 0)
  gosh> (observe (random-q))
  (0 1)
  gosh> (observe (random-q))
  (0 1)
  \end{verbatim}
  二回に一回は $\ket{0}$ が捕まえられますね。
\end{minipage}

\page{
  $\ket{1}$ が来たら
  $\left(
    \begin{array}{cc}
      0 & 1 \\
      1 & 0 \\
    \end{array}
  \right)$
  を掛けてください。
}{
  $$
  \left(
    \begin{array}{cc}
      0 & 1 \\
      1 & 0 \\
    \end{array}
  \right)
  \left(
    \begin{array}{cc}
      0\\
      1\\
    \end{array}
  \right)
=
  \left(
    \begin{array}{c}
      1 \\
      0 \\
    \end{array}
  \right)
$$
}



\hline
\begin{minipage}{.01\hsize}
  ~
\end{minipage}
\begin{minipage}{.39\hsize}
  これを Xゲート
  と言います。忘れないうちに定義しておきましょう!
\end{minipage}
\begin{minipage}{.10\hsize}
  ~
\end{minipage}
\begin{minipage}{.39\hsize}
  \begin{verbatim}
(define (X p)
  (list (ref p 0)
        (ref p 1)))
  \end{verbatim}
\end{minipage}

\page{
  qビットにはZゲートを掛けることもできます。
  Zゲートは次の行列で表現されます。
  $$
  \left( \begin{array}{cc}
    1 & 0 \\
    0 & -1
  \end{array} \right)
  $$
}{
  $$
  \left( \begin{array}{cc}
    1 & 0 \\
    0 & -1
\end{array} \right)
\left( \begin{array}{c}
    a \\ b
\end{array} \right)
  =
  \left( \begin{array}{c}
    a \\ -b
\end{array} \right)
  $$
}

\page{
  XもZもユニタリー変換です。
  ユニタリーって知ってますよね。
}{
  ユニタリー行列ならしってます。
  $$U^\dagger U = I$$
  となる行列$U$
  です。
}
\page{
  $U^\dagger$ は転置共役行列ですか。
}{そうです。}


\page{
  もっと簡単に説明できますか。
}
{
  行列$U$ の第$i$列目だけを取り出してできるベクトルを $u_i$ とすると、
  $$u_i^* u_j = \delta_{i,j}$$
  です。
}

\page{
$\delta_{i,j}$ とはクロネッカーのデルタのことですね。}
{そうです。
  $$\delta_{i,j} = \begin{cases}
    1 & i = j \text{ のとき} \\
    0 & i \ne j \text{ のとき}
  \end{cases}$$
}

\page{
  多qビットへの操作として、
  任意の $n \times n$ ユニタリー行列: $U$ でもって
  $$\ket{\varphi} \mapsto U \ket{\varphi}$$
  とできます。
}{
  任意のユニタリー行列ですか。それは驚きですね。
}

