%{(call/cc}

つまり、線路というのは、強く境界線として働くのだな.
駅が出来て、それから駅を中心に街が発展する場合、
当然線路というものが街の発展に影響を及ぼすだろう.
それがなんだか、線路という結界が街にかけられた呪いみたいで面白くて、
私はメモ帳に
「線路は境界線として強く働く」
とだけ書いてご機嫌だった.
これが私であった.
携帯電話は持たされず、
自分のことを天才だと信じ、
公園でホームレスに話しかけるのがささやかな遊戯で、
得意科目は理数系、
苦手科目は暗記全般.

あなたの趣味は何ですかと聞かれて、
一日中考えても結局趣味と言えるような趣味は無く、
まあ強いていえば読書かな、と答える.
趣味が読書と言うとなんだか無難なトコを突いてきただけだな、
と思われるのが嫌で読書と映画鑑賞だけは言いたいないのだけど、
でも本当に、そのくらいしかないので仕方がない.
自宅から予備校に行くまでの途中にある古本屋とゲームセンターはほとんど把握していた.
最近自分が発明したほとんどお金を掛けずに時間を潰せる遊び方は、
とにかく長く読めそうな古本を買ってきて、
ゲームセンターが開くまでの時間つぶしにすることだった.
予備校をサボった日はよくこれを実践した.
本は選り好みをせず漫画だろうと新書だろうと何でも読んだ.
一冊を長く時間を掛けて読めることが必要だった。

\hline

私は一人で駅を南口から入り、一旦地下を歩いて、
(小さな駅には似つかわしくないことに地下は二階分の深さがあり、
いつでも30秒で出来たてのハンバーガーを売るお店があり、
個人営業の本屋があり、
その向かいには雀荘があって、
他にも、あんまり注目がしたことがないので何なのかはよく知らないけど、
たぶん、普通に営業するお店が、こんな狭い面積に密集している)
一人で階段でホームにあがる.
一人で電車に乗った.
電車から外の風景を見下ろしていれば、すぐに私の家の最寄り駅に着く.

見下ろすっていうのは、今の場合、まったく適切な表現で、
この線はここから3駅の間、すこし高いところを走る.
下は、長い下り坂になっているが、線路はなお水平に走るためである.

一人の少女が歩いてるのを見た. そして私は、
「彼女はどうして一人なのだろう」 という議論を始めた.
「つまり、彼女は私と同じなんだ」 という説を支持することにした.
坂を下る少女が私と同じだと考えた途端、 彼女のことをかわいそうに思えた
(N.B. かわいそうな人であることに気がついた).

それはつまり、私は私のことをかわいそうなの?
という疑念は切って捨てなければならない、と自分に言い聞かせた.
他人だからこそ、かわいそうなのだ. 自己憐憫はもっとかわいそうだ.
自分をかわいそうと思うのは世界で一番かわいそうな人だけの権利でしょう.
自己憐憫は一種の快楽だ. 全く何も産まない、非生産的な快楽だ.
そんなことは、ちょっと頭を掠ったところで、五秒で止めるべきだわ.
そんな人はわざと自分で自分を貶める.
それでそんな状況を誰かにひけらかしたがる.
凛々しい猫よりも、濡れて泥に汚れたしょぼくれた猫のほうが大切に扱われると信じてる.
そんな姿、見せることを恥だと思わない. 見せられた方はたまらないと思う.
ああ、つまらないことを考えた. ようするに、ね.
私は可哀想ではないから、もっと下を見るのだ.
言い換えればもっとかわいそうな人がどこかにいるのだ.
例えば、そこにいる、 可哀想な少女は、地面にしゃがみこんでいて、
そうして、私と彼女との距離はまもなく見えないほど離れていった.
まったく可愛そうだ!

