% (k)

今日は長い夢を見た.
目が覚めるとすぐに自分は、それがいい夢だったか悪い夢だったかを思い出そうとする.
きっといい夢だったに違いない.
内容は全然覚えていないけれど、
目覚めがとってもいい気分だったから.
もっとも、その起こされ方について言えば、あまりいいものとは言えなかった.
付けっぱなしにしていたテレビからブザー音のような鈍く長い音で起きたのだった.
テレビが付けっぱなしだったのはたまたまではない.
最近はパソコンでもテレビでもなんでも付けっぱなしで寝る癖がついてしまった.
人間の声を聞きながらでないと眠りにつけない癖がついてしまったのだ.

テレビではそのブザー音の後に男のニュースキャスターらしい人が
なんだか気味の悪い事を言っていた気がする.
カーテンの隙間から漏れる太陽光は気持ちがよくて
(お昼すぎだったけれど)、
部屋のこもった空気も新鮮なものに思われた.
そのくらい気分の良い目覚めだったのだ.

公園は大体、ベビーカーを伴って談笑する若い母親、サッカーをする小学生、
それからこれは公園の奥の方で初めてこの公園に来た人までは来ないであろう
場所にホームレスの人たちが数人、ベンチで煙草を吸っている.
公園までの道自体もなかなか小洒落でジョギングなんかがよく似合う.
駅前広場に戻ろう.
アスファルトを丸く切り取って一本の太い木が植えられている.
それを取り囲むようにドーナツ型のベンチがあり、老人と子供たちが占拠している.
あの子供たちはきっと、家で遊んでいたところをお母さんに追い出されたに違いないのだ.
「せっかく天気もいいのだから、みんなで外で遊んできなさい」\\
だとか言われて.
結局子供たちはこんな所でカードゲームをしているだけなのに.
紙のカードだから風で飛ばされないか心配だ.
広場の右方にはコンビニと文房具店、あとドラッグストアがある.
自分は家に戻る前にその文房具屋で何か買う予定だったのだ.
手のひらにボールペンで書いたメモを見て何を買おうとしていたのか、
思い出そうとした.

誰かにそのメモを見られるなんて心配はあるはずもないのに、
念のために誰かに見られても恥ずかしくないように
直接買う予定の品を書かず、頭文字をアルファベットで書いてある.
暗号は再び解読する必要に迫られた.
老人たちはファーストフード店で買ったのであろうポテトを食べながら
何か話しているが、それが楽しそうには見えない.
突然、老人のうちの一人のお婆さんがやおら立ち上がって先の子供たちに話しかけた.
その光景を見て突然、既視感に襲われた.
目から入った視覚データが神経を伝って脳に入り込んだが何かの間違いで
ラッチに入り込み、失敗して無限のループに陥った.
ぐるぐるとループする光景というデータはそれを何とか抜けだそうとし、
誤って記憶領域の中を通り抜けてこれを認識したので昔見た光景を思い出しているという錯覚、
そして今見ている光景と一致しているという非現実的な知覚をしているのだな、
という妄想をした.
それにしてもこれは不思議なことだ.
ほとんど毎日見慣れている光景なのだから、昔見た光景であるのには間違いないのに.
しかしこれは、デジャブだった.

