かくして自分はアリカワという男と暮らすことになりました.
この男は大学の教員だみたいなことを言っておりましたが、
しかし、今の今までの一ヶ月をずっとこの家に閉じこもって過ごしていたそうです
(そんなことって、できるのでしょうかね).
十分な水とカロリーメイトさえあれば生きられる、と、
さもこの世のありがたい真理であるかのように説明してみせました.
彼は、自分には大した興味を示さず書斎に篭もり、何やらしており、
自分は、じゃあまあ、料理役と掃除役を仰せつかったというわけなのです.
それはそうと、この事態の収拾を着けなければと、テレビを一日見て過ごしました.
しかし、ついぞ、テレビは愛らしい猫をただ愛でるだけの娯楽番組くらいしか、
自分が興味を示すようなものは何もありませんでした.
結局自分は何もすることがなく、その日は、材料を焼いただけという、
自分ができる精一杯の料理を披露しただけで終わりました.
(この材料は、近くのスーパーを見つけて調達してきたものですが、
電気は停まっており、ものが腐らぬうちに緊急に回収してきたものなのです.
不思議なことに、アリカワの家には電気は通っておりました.)

次の日になってもニュースは何も報じませんでした.
新聞は届きませんでした.
この街に配達する者がいないようなのでした
(もっとも、この男はそういったものを読みそうにありませんが).
居候してもらっているのに、失礼だとは思いましたが、
彼の書斎を覗いてみました.
一応、コーヒーだけ淹れて.
初め、ピンポンで遊んでいるのかと思いました
(一応言っておくと、ブラウン管の左端と右端とに沿って動く棒状のブロックでボールを弾きあう卓球ゲームのことですよ).
白い画面の真ん中に明るい緑色の正方形のブロックがあって、
よく見るとそれを黒い細い線で四角く囲ってある、
そんな画面を映しだしたパソコンを、
真面目くさった顔で見つめている男が、昨日以来、まだ言葉を交わしていない男でした
(いい忘れていましたが、男は何も言ってくれないので、
私は勝手にシャワーを借りて、勝手にテレビの前に置いてあるソファで寝たのでした).
アリカワは私の腕を左手で握って、それから盆のコーヒーカップを手に取りました.\\
「これはね、神様なんだよ」\\
意味がわかりませんでした.\\
「見ててね」\\
アリカワがキーボードのどれかを押すと、緑色の四角は真下にすーっと降りて、
やがて黒い線にぶつかりました.\\
「ただ壁にぶつかることを繰り返して、やがて死ぬ」\\
別なキーを押すと、今度はテキストの羅列が表示されました.\\
「自分の幸福を追及させただけだというのに、
自分の体力を奪う行動しかしない」\\
「その神様は、あの、まず、どうして神様?」\\
「呼び方なんてのはなんだっていいんだが、まあ、そうだな。この完結した世界にただ一体だけある生き物だから、ってことでいいかな」\\
「いいんですけど、呼びにくいかな」\\
「じゃあBと呼んでくれればよいよ」\\
「どうしてB?」\\
「Bという名前の構造体で定義したから」\\
「わかりました。じゃあ、そのBはどんな行動が取れるのですか? あるいは、外部からBへの影響を及ぼすことはありますか? 」\\
「まず、外部からの影響というものはない。唯一動けるものがBだから。しかし作用反作用はある。
つまり、Bが壁にぶつかると衝撃がBと壁との両方に加わる。
壁は世界の端っこを意味する。端っこを壁だと定義しているから、これが壊れたりなどということはない。
Bが取れる行動は、この壁の内側を自由に移動するだけだ。
連続的でさえあれば速度は自由に調整できる。」\\
思わず、「だから?」と聞き返しそうになったが、やめた.
変に飽きてしまって、退屈してしまったら、気が変になる.
ここは一つ、とことん付き合ってあげなくちゃ!

